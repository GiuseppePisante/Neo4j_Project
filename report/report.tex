\documentclass[a4paper,12pt]{article}
\usepackage[english]{babel}
\usepackage{graphicx}
\usepackage{tikz}
\usepackage{wrapfig}
\usepackage{array}
\usepackage{color} 
\usepackage{hyperref}
\usepackage{enumitem}
\usepackage{booktabs}
\usepackage[font=small,labelfont=bf]{caption}
\hypersetup{
    colorlinks,
    citecolor=black,
    filecolor=black,
    linkcolor=black,
    urlcolor=black
}
\usepackage{changepage}
\addto{\captionsenglish}{\renewcommand{\refname}{}}

\begin{document}

\title{%
  Group Project - Neo4j \\
  \large of Systems and Methods for Big
    and Unstructured Data Course \\(SMBUD)\\
    held by\\ Brambilla Marco\\ Tocchetti Andrea \\
  \vspace{5mm}
  \Large \textbf{Group 78}}
\author{Pisante Giuseppe\\
  \texttt{10696936}
  \and
  Raffaelli Martina\\
  \texttt{10709893}
}
\date{Academic year 2024/2025}
\maketitle
\begin{center}
  \includegraphics[width=4cm]{polilogo.png}\\
\end{center}
\newpage
\tableofcontents
\newpage
\section{Introduction}
\paragraph{} The project aims to design and implement a database system to support the management of data related to the film industry. The database will include entities such as Person (actor, director, writer), Title (film, TV series), Episode, ratings, and genre. The goal is to create a comprehensive system that can store and query information about films, TV series, and the people involved in their production.
The project will develop with Neo4j, to exploit the relations between the entities and find significant insights from the data, such as collaborations between directors and actors, trend of genres over time, most working actors and the most successful films.
\section{Assumptions}
\paragraph{} The project is based on the following assumptions:
\begin{itemize}[noitemsep]
   \item[-] Each person has a unique ID and a name, surname, and date of birth. Some could also have a death date, if passed away.
   \item[-] Each person can be associated with multiple roles (actor, director, writer, archive footage, music department, producer)
   \item[-] Each title has a unique identifier and a title type (film, TV series, shortfilm)
   \item[-] Each title can have multiple episodes
   \item[-] Each title can have multiple genres
   \item[-] Each user can rate a title if and only if it has watched the film
   \item[-] Each title has a unique rating, average rating from the users, and the number of votes
   \item[-] Each person can be associated with multiple titles
   \item[-] Each title can have multiple people associated with it
\end{itemize}
\clearpage
\section{ER diagram}
\paragraph{}
	\begin{center}
 		\includegraphics[width = 15 cm]{polilogo.png}
		\captionof{figure}{E-R Diagram}
	\end{center}
  \subsection{Entities}
\par Starting from the considerations previously exposed regarding the implementation hypotheses, we have drawn an ER diagram (\textbf{Figure 1}) which includes 5 different entities and 7 many-to-many relationships described below in the logical model: \par
  \begin{itemize}[noitemsep]
  \item[-]	\textbf{Person}(\underline{nconst}, PrimaryName, BirthYear, DeathYear, PrimaryProfession, KnownForTitles)
	\item[-]	\textbf{Title}(\underline{tconst}, PrimaryTitle, OriginalTitle, TitleType, StartYear, EndYear, RuntimeMinutes, Genres)
	\item[-]	\textbf{Episode}(\underline{tconst}, ParentTconst, SeasonNumber, EpisodeNumber)
	\item[-]	\textbf{Ratings}(\underline{tconst}, AverageRating, NumVotes)
	\item[-]	\textbf{Genre}(Name)
  \end{itemize} \par
The \textbf{Person} entity describes every possible individual with their own personal data, including their primary profession and titles they are known for. 
The \textbf{Title} entity represents films, TV series or short films with their respective attributes. 
The \textbf{Episode} entity is used to detail episodes of TV series, linked to their parent series. 
The \textbf{Ratings} entity captures the average rating and number of votes for each title. 
Finally, the \textbf{Genre} entity categorizes the titles into different genres.

\subsection{Relationships}
\begin{itemize}[noitemsep]
  
    \item[\textbf{ACTED\_IN}] $(Person)-[:ACTED\_IN]->(Title)$
    
    Relationship between a Person, whose primary profession is actor, and a Title.
    \item[\textbf{DIRECTED}] $(Person)-[:DIRECTED]->(Title)$
    
    Relationship between a Person, whose primary profession is director, and a Title.
    \item[\textbf{WROTE}] $(Person)-[:WROTE]->(Title)$
    
    Relationship between a Person, whose primary profession is writer, and a Title.
    \item[\textbf{PART\_OF}] $(Episode)-[:PART\_OF]->(Title)$
    
    Relationship between an Episode and its parent Title, whose TitleType is TV series.
    \item[\textbf{HAS\_GENRE}] $(Title)-[:HAS\_GENRE]->(Genre)$
    
    Relationship between a Title and a Genre.
    \item[\textbf{HAS\_RATING}] $(Title)-[:HAS\_RATING]->(Rating)$
    
    Relationship between a Title and its Rating.

    
\end{itemize}


\subsection{Constraints:}
ciao

% \section{Import database}
% ciao
\section{Cypher Queries}
\subsection{DistributionCenter closest to each User}
\begin{verbatim}
  MATCH (u:User), (d:DistributionCenter)
  WITH u, d, distance(point({latitude: u.latitude, longitude: u.longitude}),
                      point({latitude: d.latitude, longitude: d.longitude})) AS dist
  RETURN u.id AS user_id, d.name AS nearest_center, dist
  ORDER BY dist ASC
  LIMIT 10;
\end{verbatim}
\begin{table}[h!]
  \centering
  \begin{tabular}{lll}
  \toprule
  \textbf{user\_id} & \textbf{nearest\_center} & \textbf{dist} \\ 
  \midrule
  58210 & Chicago IL & 1759.9113 \\ 
  48442 & Chicago IL & 1759.9113 \\ 
  20571 & Chicago IL & 1759.9113 \\ 
  37767 & Philadelphia PA & 1897.4164 \\ 
  7543  & Philadelphia PA & 1897.4164 \\ 
  41572 & Philadelphia PA & 1897.4164 \\ 
  \bottomrule
  \end{tabular}
  \caption{Output della query Neo4j.}
  \label{tab:query_output}
  \end{table}

  \subsection{Most sold products:}
  \begin{verbatim}  
    MATCH (:OrderItem)-[:REFERS_TO]->(p:Product)
    RETURN p.name AS product_name, count(*) AS sales_count
    ORDER BY sales_count DESC
    LIMIT 10;
  \end{verbatim}
  \begin{table}[h!]
    \centering
    \begin{tabular}{p{8cm}r} % Colonna larga per i nomi e destra per i numeri
    \toprule
    \textbf{Product Name} & \textbf{Quantity} \\ 
    \midrule
    Wrangler Men's Premium Performance Cowboy Cut Jean & 62 \\
    Puma Men's Socks & 48 \\
    7 For All Mankind Men's Standard Classic Straight Leg Jean & 41 \\
    True Religion Men's Ricky Straight Jean & 37 \\
    Kenneth Cole Men's Straight Leg Jean & 36 \\
    Michael Kors Men's 3 Pack Brief & 33 \\
    HUGO BOSS Men's Long Pant & 31 \\
    Lucky Brand Mens Men's 361 Vintage Straight Denim Jean & 31 \\
    Thorlo Unisex Experia Running Sock & 31 \\
    Diesel Men's Shioner Skinny Straight Leg Jean & 31 \\
    \bottomrule
    \end{tabular}
    \caption{Top-selling men's clothing products and their quantities.}
    \label{tab:top_selling_products}
    \end{table}






\newpage
\section{References \& Sources}
  \begin{thebibliography}{9}
    \bibitem{} Course Slides
    \bibitem{} https://pysimplegui.readthedocs.io/en/latest/call%20reference/
    \bibitem{} https://py2neo.org/
    \bibitem{} https://neo4j.com/docs/cypher-manual/current/
    \bibitem{} https://neo4j.com/developer/python/
    \bibitem{} http://iniball.altervista.org/Software/ProgER
    \bibitem{} https://neo4j.com/developer/cypher/
    \bibitem{} https://pandas.pydata.org/docs/
  \end{thebibliography}
\end{document}