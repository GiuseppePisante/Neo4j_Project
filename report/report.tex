\documentclass[a4paper,12pt]{article}
\usepackage[english]{babel}
\usepackage{graphicx}
\usepackage{tikz}
\usepackage{wrapfig}
\usepackage{array}
\usepackage{color} 
\usepackage{hyperref}
\usepackage{enumitem}
\usepackage[font=small,labelfont=bf]{caption}
\hypersetup{
    colorlinks,
    citecolor=black,
    filecolor=black,
    linkcolor=black,
    urlcolor=black
}
\usepackage{changepage}
\addto{\captionsenglish}{\renewcommand{\refname}{}}

\begin{document}

\title{%
  Group Project - Neo4j \\
  \large of Systems and Methods for Big
    and Unstructured Data Course \\(SMBUD)\\
    held by\\ Brambilla Marco\\ Tocchetti Andrea \\
  \vspace{5mm}
  \Large \textbf{Group 78}}
\author{Pisante Giuseppe\\
  \texttt{10696936}
  \and
  Raffaelli Martina\\
  \texttt{10709893}
}
\date{Academic year 2024/2025}
\maketitle
\begin{center}
  \includegraphics[width=4cm]{polilogo.png}\\
\end{center}
\newpage
\tableofcontents
\newpage
\section{Introduction}
\paragraph{} The project aims to design and implement a database system to support the management of data related to the film industry. The database will include entities such as Person (actor, director, writer), Title (film, TV series), Episode, ratings, and genre. The goal is to create a comprehensive system that can store and query information about films, TV series, and the people involved in their production.
The project will develop with Neo4j, to exploit the relations between the entities and find significant insights from the data, such as collaborations between directors and actors, trend of genres over time, most working actors and the most successful films.
\section{Assumptions}
\paragraph{} The project is based on the following assumptions:
\begin{itemize}[noitemsep]
   \item[-] Each person has a unique ID and a name, surname, and date of birth. Some could also have a death date, if passed away.
   \item[-] Each person can be associated with multiple roles (actor, director, writer, archive footage, music department, producer)
   \item[-] Each title has a unique identifier and a title type (film, TV series, shortfilm)
   \item[-] Each title can have multiple episodes
   \item[-] Each title can have multiple genres
   \item[-] Each user can rate a title if and only if it has watched the film
   \item[-] Each title has a unique rating, average rating from the users, and the number of votes
   \item[-] Each person can be associated with multiple titles
   \item[-] Each title can have multiple people associated with it
\end{itemize}
\clearpage
\section{ER diagram}
\paragraph{}
	\begin{center}
 		\includegraphics[width = 15 cm]{polilogo.png}
		\captionof{figure}{E-R Diagram}
	\end{center}
  \subsection{Entities}
\par Starting from the considerations previously exposed regarding the implementation hypotheses, we have drawn an ER diagram (\textbf{Figure 1}) which includes 5 different entities and 7 many-to-many relationships described below in the logical model: \par
  \begin{itemize}[noitemsep]
  \item[-]	\textbf{Person}(\underline{nconst}, PrimaryName, BirthYear, DeathYear, PrimaryProfession, KnownForTitles)
	\item[-]	\textbf{Title}(\underline{tconst}, PrimaryTitle, OriginalTitle, TitleType, StartYear, EndYear, RuntimeMinutes, Genres)
	\item[-]	\textbf{Episode}(\underline{tconst}, ParentTconst, SeasonNumber, EpisodeNumber)
	\item[-]	\textbf{Ratings}(\underline{tconst}, AverageRating, NumVotes)
	\item[-]	\textbf{Genre}(Name)
  \end{itemize} \par
The \textbf{Person} entity describes every possible individual with their own personal data, including their primary profession and titles they are known for. 
The \textbf{Title} entity represents films, TV series or short films with their respective attributes. 
The \textbf{Episode} entity is used to detail episodes of TV series, linked to their parent series. 
The \textbf{Ratings} entity captures the average rating and number of votes for each title. 
Finally, the \textbf{Genre} entity categorizes the titles into different genres.

\subsection{Relationships}
\begin{itemize}[noitemsep]
  
    \item[\textbf{ACTED\_IN}] $(Person)-[:ACTED\_IN]->(Title)$
    
    Relationship between a Person, whose primary profession is actor, and a Title.
    \item[\textbf{DIRECTED}] $(Person)-[:DIRECTED]->(Title)$
    
    Relationship between a Person, whose primary profession is director, and a Title.
    \item[\textbf{WROTE}] $(Person)-[:WROTE]->(Title)$
    
    Relationship between a Person, whose primary profession is writer, and a Title.
    \item[\textbf{PART\_OF}] $(Episode)-[:PART\_OF]->(Title)$
    
    Relationship between an Episode and its parent Title, whose TitleType is TV series.
    \item[\textbf{HAS\_GENRE}] $(Title)-[:HAS\_GENRE]->(Genre)$
    
    Relationship between a Title and a Genre.
    \item[\textbf{HAS\_RATING}] $(Title)-[:HAS\_RATING]->(Rating)$
    
    Relationship between a Title and its Rating.

    
\end{itemize}


%\subsection{Constraints:}
%\begin{itemize}[noitemsep]
%    \item \textbf{Unique Constraints:}
%    \begin{itemize}
%        \item \texttt{CREATE CONSTRAINT FOR (p:Person) REQUIRE p.nconst IS UNIQUE}
%        \item \texttt{CREATE CONSTRAINT FOR (t:Title) REQUIRE t.tconst IS UNIQUE}
%        \item \texttt{CREATE CONSTRAINT FOR (e:Episode) REQUIRE (e.tconst, e.ParentTconst,}
%         \texttt{e.SeasonNumber, e.EpisodeNumber) IS UNIQUE}
%        \item \texttt{CREATE CONSTRAINT FOR (r:Ratings) REQUIRE r.tconst IS UNIQUE}
%        \item \texttt{CREATE CONSTRAINT FOR (g:Genre) REQUIRE g.Name IS UNIQUE}
%    \end{itemize}
%    
%    \item \textbf{Foreign Key Constraints:}
%    \begin{itemize}
%        \item \texttt{CREATE CONSTRAINT FOR (e:Episode)-[:PART_OF]->(t:Title)}
%        \texttt{REQUIRE e.ParentTconst = t.tconst}
%        \item \texttt{CREATE CONSTRAINT FOR (r:Ratings)-[:RATED]->(t:Title) REQUIRE r.tconst = t.tconst}
%        \item \texttt{CREATE CONSTRAINT FOR (a:ACTED_IN)-[:ACTED_IN]->(p:Person, t:Title)}
%        \texttt{REQUIRE a.Person = p.nconst AND a.Title = t.tconst}
%        \item \texttt{CREATE CONSTRAINT FOR (d:DIRECTED)-[:DIRECTED]->(p:Person, t:Title)}
%        \texttt{REQUIRE d.Person = p.nconst AND d.Title = t.tconst}
%        \item \texttt{CREATE CONSTRAINT FOR (w:WROTE)-[:WROTE]->(p:Person, t:Title)}
%        \texttt{REQUIRE w.Person = p.nconst AND w.Title = t.tconst}
%        \item \texttt{CREATE CONSTRAINT FOR (e:Episode)-[:PART_OF]->(t:Title)}
%        \texttt{REQUIRE e.ParentTconst = t.tconst}
%        \item \texttt{CREATE CONSTRAINT FOR (h:HAS_GENRE)-[:HAS_GENRE]->(t:Title, g:Genre)}
%        \texttt{REQUIRE h.Title = t.tconst AND h.Genre = g.Name}
%    \end{itemize}
%    
%    \item \textbf{Other Constraints:}
%    \begin{itemize}
%        \item \texttt{CREATE CONSTRAINT FOR (t:Title) REQUIRE t.StartYear <= t.EndYear}
%        \item \texttt{CREATE CONSTRAINT FOR (e:Episode) REQUIRE e.SeasonNumber > 0 AND e.EpisodeNumber > 0}
%        \item \texttt{CREATE CONSTRAINT FOR (r:Ratings) REQUIRE r.AverageRating >= 0 AND r.AverageRating <= 10}
%        \item \texttt{CREATE CONSTRAINT FOR (r:Ratings) REQUIRE r.NumVotes >= 0}
%    \end{itemize}
%\end{itemize}

\section{Import database}

\section{Cypher Queries}
\subsection{Trova tutti gli attori che hanno lavorato in un titolo con un determinato genere}
MATCH (p:Person)-[:ACTED_IN]->(t:Title)-[:HAS_GENRE]->(g:Genre)
WHERE g.Name = 'Drama'
RETURN p.PrimaryName, t.PrimaryTitle

\subsection{Identifica i registi che hanno diretto il maggior numero di titoli}
MATCH (p:Person)-[:DIRECTED]->(t:Title)
RETURN p.PrimaryName, COUNT(t) AS TitlesDirected
ORDER BY TitlesDirected DESC
LIMIT 10

\subsection{Trova tutti i titoli con una valutazione superiore a 8 e con più di 10.000 voti}
MATCH (t:Title)-[:HAS_RATING]->(r:Ratings)
WHERE r.AverageRating > 8 AND r.NumVotes > 10000
RETURN t.PrimaryTitle, r.AverageRating, r.NumVotes
ORDER BY r.AverageRating DESC

\subsection{Elenca tutti i generi di un titolo specifico}
MATCH (t:Title)-[:HAS_GENRE]->(g:Genre)
WHERE t.PrimaryTitle = 'Inception'
RETURN g.Name

\subsection{Trova i migliori titoli per ogni genere}
MATCH (t:Title)-[:HAS_GENRE]->(g:Genre), (t)-[:HAS_RATING]->(r:Ratings)
WITH g.Name AS Genre, t.PrimaryTitle AS Title, r.AverageRating AS Rating
ORDER BY Genre, Rating DESC
RETURN Genre, Title, Rating

\subsection{Esplora gli attori che hanno lavorato frequentemente con lo stesso regista}
MATCH (d:Person)-[:DIRECTED]->(t:Title)<-[:ACTED_IN]-(a:Person)
RETURN d.PrimaryName AS Director, a.PrimaryName AS Actor, COUNT(t) AS Collaborations
ORDER BY Collaborations DESC
LIMIT 10

\subsection{Trova tutti gli episodi di una determinata serie TV}
MATCH (e:Episode)-[:PART_OF]->(t:Title)
WHERE t.PrimaryTitle = 'Breaking Bad'
RETURN e.SeasonNumber, e.EpisodeNumber, e.tconst
ORDER BY e.SeasonNumber, e.EpisodeNumber

\subsection{Determina il titolo con il maggior numero di episodi}
MATCH (e:Episode)-[:PART_OF]->(t:Title)
RETURN t.PrimaryTitle, COUNT(e) AS EpisodeCount
ORDER BY EpisodeCount DESC
LIMIT 1

\subsection{Trova tutte le serie TV con una durata media di episodio maggiore di 50 minuti}
MATCH (e:Episode)-[:PART_OF]->(t:Title)
WHERE t.TitleType = 'tvSeries' AND t.RuntimeMinutes > 50
RETURN t.PrimaryTitle, t.RuntimeMinutes

\subsection{Identifica gli attori più prolifici}
MATCH (p:Person)-[:ACTED_IN]->(t:Title)
RETURN p.PrimaryName, COUNT(t) AS NumberOfTitles
ORDER BY NumberOfTitles DESC
LIMIT 10

\subsection{Trova i titoli diretti, scritti e recitati dalla stessa persona}
MATCH (p:Person)-[:DIRECTED]->(t:Title)<-[:WROTE]-(p)-[:ACTED_IN]->(t)
RETURN p.PrimaryName, t.PrimaryTitle

\subsection{Trova i titoli senza rating}
MATCH (t:Title)
WHERE NOT (t)-[:HAS_RATING]->()
RETURN t.PrimaryTitle

\subsection{Analizza la distribuzione dei generi per ogni attore}
MATCH (p:Person)-[:ACTED_IN]->(t:Title)-[:HAS_GENRE]->(g:Genre)
RETURN p.PrimaryName, g.Name, COUNT(t) AS NumberOfTitles
ORDER BY p.PrimaryName, NumberOfTitles DESC

\subsection{Trova i registi che lavorano in più generi}
MATCH (p:Person)-[:DIRECTED]->(t:Title)-[:HAS_GENRE]->(g:Genre)
RETURN p.PrimaryName, COLLECT(DISTINCT g.Name) AS Genres, COUNT(g) AS TotalGenres
ORDER BY TotalGenres DESC

\subsection{Trova i titoli con la durata massima}
MATCH (t:Title)
WHERE t.RuntimeMinutes IS NOT NULL
RETURN t.PrimaryTitle, t.RuntimeMinutes
ORDER BY t.RuntimeMinutes DESC
LIMIT 10

\subsection{Identifica gli attori più votati}
MATCH (p:Person)-[:ACTED_IN]->(t:Title)-[:HAS_RATING]->(r:Ratings)
RETURN p.PrimaryName, SUM(r.NumVotes) AS TotalVotes
ORDER BY TotalVotes DESC
LIMIT 10

\subsection{Trova gli episodi con la migliore valutazione di una serie specifica}
MATCH (e:Episode)-[:PART_OF]->(t:Title)-[:HAS_RATING]->(r:Ratings)
WHERE t.PrimaryTitle = 'The Office'
RETURN e.SeasonNumber, e.EpisodeNumber, r.AverageRating
ORDER BY r.AverageRating DESC

\subsection{Trova tutte le persone che hanno collaborato in almeno 5 titoli}
MATCH (p:Person)-[:ACTED_IN|DIRECTED|WROTE]->(t:Title)
WITH p, COUNT(DISTINCT t) AS CollaborationCount
WHERE CollaborationCount >= 5
RETURN p.PrimaryName, CollaborationCount

\subsection{Trova le serie TV più popolari (somma dei voti degli episodi)}
MATCH (e:Episode)-[:PART_OF]->(t:Title)-[:HAS_RATING]->(r:Ratings)
WHERE t.TitleType = 'tvSeries'
RETURN t.PrimaryTitle, SUM(r.NumVotes) AS TotalVotes
ORDER BY TotalVotes DESC
LIMIT 10

\subsection{Analizza i registi più votati per i loro titoli}
MATCH (p:Person)-[:DIRECTED]->(t:Title)-[:HAS_RATING]->(r:Ratings)
RETURN p.PrimaryName, SUM(r.NumVotes) AS TotalVotes, AVG(r.AverageRating) AS AvgRating
ORDER BY TotalVotes DESC
LIMIT 10




\newpage
\section{References \& Sources}
  \begin{thebibliography}{9}
    \bibitem{} Course Slides
    \bibitem{} https://pysimplegui.readthedocs.io/en/latest/call%20reference/
    \bibitem{} https://py2neo.org/
    \bibitem{} https://neo4j.com/docs/cypher-manual/current/
    \bibitem{} https://neo4j.com/developer/python/
    \bibitem{} http://iniball.altervista.org/Software/ProgER
    \bibitem{} https://neo4j.com/developer/cypher/
    \bibitem{} https://pandas.pydata.org/docs/
  \end{thebibliography}
\end{document}